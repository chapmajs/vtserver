\documentstyle[palatino,mya4]{article}[12pt]
\setlength{\parindent}{0pt}
\setlength{\parskip}{4pt}
\begin{document}

\begin{center}
{\LARGE \bf VTserver: Installing a PDP-11 UNIX with No Tape Drive}

{\Large \bf Warren Toomey}

{\large \it wkt@cs.adfa.edu.au}

{\large Version 2.3, March 2001}
\end{center}


\section{Introduction}

The Vtserver program provides you with a method of copying a disk image
into a PDP-11, or extracting a disk image from a PDP-11, without a tape
drive or indeed an installed operating system.

The approach here is use a nearby Unix or Linux computer as both the
PDP-11 console and as a virtual tape server.
The VTserver software comes as two components: a set of PDP-11
software which acts as the virtual tape client, and the server which is hosted
on the nearby Unix machine.

The two computers are connected via an RS-232 null modem with hardware
handshaking. A KL/DL-11 unit on the PDP-11 is used as the serial port: the
highest transfer rate is 9,600 bps [I could be wrong here -- please
let me know otherwise]. The main limitation at present is that the PDP-11
must have at least 192 Kbytes of memory.

\section{Manifest}

The VTserver software is available from
ftp://minnie.cs.adfa.edu.au/pub/PDP-11/Vtserver as the file \\
{\tt vtserver2.3x-yyyymmdd.tar.gz}, where the {\it x-yyyymmdd}
represents the sub-version and the date of release: it's always best
to get the latest version.

When you extract the files from this tar archive, you will find
directories {\it vtserver/} and {\it pdpvtstand/}.
For now, the {\it vtserver/} directory is where you want to go, because
this contains the PDP-11 binaries and the server program you need.
You should have received the following files as part of the VTserver
software:

\begin{description}
\item[vtreadme.txt:] This document in plain ASCII. There may be translations
	of this document in other formats, too.
\item[vtserver.c:] The source code to the server program.
\item[.vtrc:] The server's configuration file.
\item[vtboot.pdp:] The machine code to bootstrap using the {\tt vtserver}.
\item[copy:] The standalone copying program.
\end{description}

You will also need a disk image for your PDP-11, one that is known to work for
your model. If you are trying to install a UNIX, then I recommend that you use
one of the available PDP-11 emulators plus the versions of UNIX in the UNIX
Archive to create one of these. Don't bother trying to install a disk image
unless you know that your disk image will work.

Alternatively, you may want to install 2.11BSD using the standard tape
installation procedure. In this case you will also find modified versions
of {\it boot}, {\it disklabel}, {\it mkfs}, {\it icheck} and {\it restor}.
You will need to obtain the 2.11BSD root filesystem dump and the 2.11BSD
{\it /usr} tar archives from elsewhere.

In the following section, we will discuss how to copy an existing disk
image from the virtual tape server to your disk. Towards the end, there is
a section on installing 2.11BSD using the standard tape installation procedure.

\section{Compiling {\tt vtserver}}

The first thing to do is to compile {\tt vtserver.c}:

\begin{verbatim}
  $ cc -o vtserver vtserver.c
\end{verbatim}

The code is pretty vanilla ANSI C. The only system-specific sections are that
to set the serial port to raw mode, and the use of {\it select()}. I have
used the POSIX {\it termios} system calls, and you may need to modify the
server code if your system isn't POSIX-compatible. Please send patches back to
me for this.

\section{Connecting the PDP-11 to the {\tt vtserver}}

The console KL-11 unit 
needs to be connected via RS-232 to a serial port on the Unix server where
you compiled {\tt vtserver.c}. Ok, I have no experience here as I don't have a
real PDP-11. I'd strongly recommend using a null modem cable with full hardware
handshaking lines (RTS/CTS and DTR/DSR) -- however, I have been told that
the console line doesn't support hardware handshaking. Can anybody clear
this up for me?

I'd also recommend having a copy of Kermit on the Unix server end. If no
communication seems to be taking place, open the serial line with Kermit,
and see if characters are being received from the PDP-11. An RS-232
breakout box here could also be useful.

\section{The {\tt .vtrc} File}

The file {\tt .vtrc} is the server's configuration file. Lines beginning
with hashes are ignored. The first (non-hashed) line gives a shell command
to set the correct speed and parity of the serial line. The second line names
the serial device. Remaining lines name the fictitious tape's records, and
should be {\it copy} and the name of your disk image.

An example {\tt .vtrc} would be:

\begin{verbatim}
  #       Configuration file for Virtual Tape Server
  #
  # First line is any Unix command needed to set the serial port up
  stty -f /dev/ttyid1 9600 cs8 clocal -crtscts
  #
  # Second line in this file is the serial port device
  /dev/ttyd1
  #
  # After that is the list of files on the virtual tape, e.g
  #   Record 0: the copy ptogram
  #   Record 1: a disk image for your root disk
  #
  copy
  disk.img
\end{verbatim}

When the {\tt vtserver} opens a file up as a tape record, it will first
try to do so read/write. If this fails, it will then try to open
the file read-only. And if this fails, {\tt vtserver} will try to create
the file. You will see messages from {\tt vtserver} about how a file has been
opened. The file create option allows you to name several non-existent
files in the {\tt .vtrc}; they can then be used as tape records to write
to as required.

\medskip

In your {\tt .vtrc}, you will most likely need to change the name of the
serial port device and the shell command to set it up.
You should configure the KL-11 for 9600 baud, 8 bits, no parity, and
you should do the same for the Unix server. On my FreeBSD box, the
shell command in my {\tt .vtrc} is:

\begin{verbatim}
  stty -f /dev/ttyid1 9600 cs8 clocal -crtscts
\end{verbatim}

If you are using hardware handshaking lines (and you should), change this to:

\begin{verbatim}
  stty -f /dev/ttyid1 9600 cs8 -clocal crtscts
\end{verbatim}

\section{Starting {\tt vtserver}}

{\tt vtserver} is started with no arguments, and reads its configuration
file {\tt .vtrc}. The {\tt vtserver} then gives a short description of
the tape's contents:

\begin{verbatim}
  $ vtserver
  Virtual tape server, $Revision: 1.4 $ 
  Running command stty -f /dev/ttyid1 9600 cs8 clocal -crtscts

  Tape records are:
     0 boot
     1 copy
     2 disk.img

  Opening port /dev/ttyd1 .... Port open
\end{verbatim}

At this point, you can use {\tt vtserver} as your PDP-11's console. However,
when the VT client starts sending VT commands to {\tt vtserver}, you will
see messages from the server describing which file is being transferred.

As each file is opened, its name is shown on the screen. An `r' signifies a
read of 512 bytes. If the VT client on the PDP-11 tries to read past the
file's end, an EOF is shown. Finally, a progress message is shown every 100K
of a file's download.

When not downloading a file, you can exit {\tt vtserver} by typing two
consecutive ESC keys at the console.

\section{Booting via {\tt vtserver}: Hand Method}

The first stage of the boot process is to enter the boot code which
retrieves the file {\tt boot} from the {\tt vtserver}. If you don't
have ODT in your PDP-11, then you will have to hand-enter this code.
The code should be entered at address 140000 onwards. The code is given in
the file {\tt vtboot.pdp}, and also in octal below.

\begin{verbatim}
0140000: 010706 005003 012701 177560 012704 140106 112400 100406
0140020: 105761 000004 100375 110061 000006 000770 005267 000052
0140040: 004767 000030 001403 012703 006400 005007 012702 001000
0140060: 004767 000010 110023 005302 001373 000746 105711 100376
0140100: 116100 000002 000207 025037 000000 000000 177777
\end{verbatim}

Toggle in the code, and start execution at address 140000. If all goes well,
the bootstrap code will start to send VT commands, and you will see the
following from the {\tt vtserver}:

\begin{verbatim}
  Opened copy read-write
   rrrrrrrrrrrrrrrrrrrrrrrrrrrrrrrrrrrrrrrrr EOF
  Standalone copy program - known devices are 
  ht tm ts ram hk ra rk rl rx si xp br tms vt 
\end{verbatim}

\section{Booting via {\tt vtserver}: ODT Method}

If your PDP-11 does have ODT, then you can get the {\tt vtserver} itself to
send the bootstrap code down. Firstly make sure that the ODT prompt is
awaiting a command, then (shutdown and) start {\tt vtserver} as follows:

\begin{verbatim}
  $ vtserver -odt
  Virtual tape server, $Revision: 1.4 $ 
  . . .
  Opened copy read-write
   rrrrrrrrrrrrrrrrrrrrrrrrrrrrrrrrrrrrrrrrr EOF
\end{verbatim}

The {\tt -odt} flag tells {\tt vtserver} to download the bootstrap code
and to jump to location 140000. The first tape record is downloaded,
and the standalone {\it copy} program starts up.

\begin{quote}
{\bf Note!} I don't have a PDP-11 with ODT, so this feature is untested.
If you have such a system, please try this out and let me know if it
works. The actual code to deal with ODT in {\tt vtserver.c} is relatively
simple; if you find a bug, can you fix it and send me back the changes.
\end{quote}

\section{Copying Disk Images}

When {\it copy} has loaded and starts running, you will be prompted for the
input and output devices.

\begin{verbatim}
  Standalone copy program - known devices are 
  ht tm ts ram hk ra rk rl rx si xp br tms vt 

  Tape record n from device xx is written as xx(0,0,n)
  Disk device xx is written as xx(0,0,0)

  Enter name of input record/device:
  Enter name of output record/device:
\end{verbatim}

{\it Copy} use the 2.11BSD terminology for devices, units and partitions,
and you should probably read the 2.11BSD setup documentation here. However,
to describe record {\it n} from tape device {\it xx}, you would say
{\it xx(0,0,n)}. And to describe the whole of disk device {\it xx}, you would
just say {\it xx(0,0,0)}.

Remember that you can read from any device and write to any device. So you
can write a disk image from the {\tt vtserver} to your PDP-11, or you can
copy an existing disk image from your PDP-11 to a file on the {\tt vtserver}.

The available tape and disk devices are:

\begin{description}
\item[br] RP03-like disk driver, modified to handle BR 1537 and 1711
	controllers with T300, T200, T80 and T50 drives.
\item[hk] RK06/07 disk driver.
\item[ra] MSCP disk device driver (rx23, rx33, rx50, rd??, ra??, rz??).
\item[rk] RK disk driver.
\item[rl] RL disk driver.
\item[rx] RX02 disk driver.
\item[si] SI 9500 CDC 9766 disk driver.
\item[xp] SMD disk driver.

\item[vt] Virtual Tape drive.
\item[ht] TU16/TE16/TU77 tape driver.
\item[tm] TU10/TE10/TS03 tape driver.
\item[ts] TS11/TU80/TS05/TK25 tape driver.
\end{description}

\section{An Example Disk Image Copy}

Here is an example output where the user has booted from the {\tt vtserver},
loaded {\it copy} and then copied the image {\it disk.img} to their RL02 disk.

\begin{verbatim}
  Virtual tape server, $Revision: 1.4 $ 
  Running command stty -f /dev/ttyid1 19200 cs8 clocal -crtscts

  Tape records are:
     0 copy
     1 disk.img

  Opening port /dev/ttyd1 .... Port open

  Opened copy read-write
   rrrrrrrrrrrrrrrrrrrrrrrrrrrrrrrrrrrrrrrrr EOF
  Standalone copy program - known devices are 
  ht tm ts ram hk ra rk rl rx si xp br tms vt 

  Tape record n from device xx is written as xx(0,0,n)
  Disk device xx is written as xx(0,0,0)

  Enter name of input record/device: vt(0,0,1)

  Opened disk.img read-write
   Enter name of output record/device: rl(0,0,0)
  rrrrrrrrrrrrrrrrrrrrrrrrrrrrrrrrrrrrrrrrrrrrr ... rrrr EOF
  EOF
  Copying done. Either reset the system, or hit
  <return> to exit the standalone program.
\end{verbatim}

At this point, the PDP-11 can be rebooted. If the downloaded disk image
is bootable, the user should now have a working PDP-11 operating system.

\section{A Warning}

Jay Jaeger sends in this timely warning about the size of disk images:

\begin{quote}
Something I thought is worth warning folks about.  For RL01, RL02, RK06 and 
RK07 (at least), the last cylinder is {\bf RESERVED}
for bad blocks and the pack serial number.

Except for maintaining the bad block list, software should not write there, 
and should at least preserve the serial number (and its mirror copies that 
are usually present) when doing so.

The current implementation of VTServer (in particular, the code in the ``hk'' 
driver in {\tt pdpvtstand/}) does not appear to take that into account.

The problem is that if you take a {\it full} pack image from a pack and then 
restore that {\it full} pack image to a different pack, you will wipe out the 
latter's bad block info (and perhaps write in some areas that are known not 
to be very good).

Traditionally, Unix variants dealt with this by insisting that you use 
error free (Suffix -EF in the DEC part number on the pack) disk 
packs.  However, if you are using real hardware you may no longer have 
those available.
\end{quote}

\section{Installing 2.11BSD with Vtserver}

To install 2.11BSD via the {\tt vtserver}, you need to set up a virtual
tape with records corresponding to the standard 2.11BSD tape. You don't
need any boostrap records on your virtual tape, and so your {\tt .vtrc}
file should look something like this (minus the comment lines):

\begin{verbatim}
  stty -f /dev/ttyid1 9600 cs8 clocal -crtscts
  /dev/ttyd1
  boot.dd
  disklabel
  mkfs
  restor
  icheck
  root.dump
  usr.tar
  otherusr.tar
\end{verbatim}

Don't use the standalone programs {\it disklabel},
{\it mkfs}, {\it icheck} and {\it restor} from standard 2.11BSD, as they
don't speak the VT protocol. Instead, use the VT-enabled programs supplied
with the {\tt vtserver}.

Apart from setting up the serial line, compiling and starting {\tt vtserver},
and entering in the VT boot code, you should be able to follow the standard
2.11BSD installation instructions. As an example, here is an example output
where a user made a root filesystem on an RL02, and then restored the
{\it root.dump} to this disk:

\begin{verbatim}
  $ vtserver 
  Virtual tape server, $Revision: 1.4 $ 
  Running command stty -f /dev/ttyid1 19200 cs8 clocal -crtscts
  
  Tape records are:
     0 boot.dd
     1 disklabel
     2 mkfs
     3 restor
     4 icheck
     5 root.dump
  
  Opening port /dev/ttyd1 .... Port open
  
  Opened boot.dd read-write
   rrrrrrrrrrrrrrrrrrrrrrrrrrrrrrrrrrrrrrrrrrrrrrr EOF
  
  45Boot from vt(0,0,0) at 0177560
  : vt(0,0,1)
  
  Opened disklabel read-write
   rrrrrrrrrrrrrrrrrrrrrrrrrrrrrrrrrrrrrrrrrrrrrrrrrrrrrrBoot: bootdev=06401
                                                               bootcsr=0177560
  disklabel
  Disk? rl(0,0,0)
  'rl(0,0,0)' is unlabeled or the label is corrupt.
  Proceed? [y/n] y
  d(isplay) D(efault) m(odify) w(rite) q(uit)? D
  d(isplay) D(efault) m(odify) w(rite) q(uit)? d
  
  type: old DEC
  disk: 
  label: DEFAULT
  flags:
  bytes/sector: 512
  sectors/track: 20
  tracks/cylinder: 2
  sectors/cylinder: 40
  cylinders: 512
  rpm: 3600
  drivedata: 0 0 0 0 0 
  
  1 partitions:
  #        size   offset    fstype   [fsize bsize]
    a: 20480 0  2.11BSD    1024 1024      # (Cyl. 0 - 511)
  
  d(isplay) D(efault) m(odify) w(rite) q(uit)? m
  modify
  d(isplay) g(eometry) m(isc) p(artitions) q(uit)? p
  modify partitions
  d(isplay) n(umber) s(elect) q(uit)? n
  Number of partitions (8 max) [1]? 2
  modify partitions
  d(isplay) n(umber) s(elect) q(uit)? s
  a b c d e f g h q(uit)? a
  sizes and offsets may be given as sectors, cylinders
  or cylinders plus sectors:  6200, 32c, 19c10s respectively
  modify partition 'a'
  d(isplay) z(ero) t(ype) o(ffset) s(ize) f(rag) F(size) q(uit)? s
  'a' size [20480]: 20000
  modify partition 'a'
  d(isplay) z(ero) t(ype) o(ffset) s(ize) f(rag) F(size) q(uit)? q
  modify partitions
  d(isplay) n(umber) s(elect) q(uit)? s
  a b c d e f g h q(uit)? b
  sizes and offsets may be given as sectors, cylinders
  or cylinders plus sectors:  6200, 32c, 19c10s respectively
  modify partition 'b'
  d(isplay) z(ero) t(ype) o(ffset) s(ize) f(rag) F(size) q(uit)? o
  'b' offset [0]: 20000
  modify partition 'b'
  d(isplay) z(ero) t(ype) o(ffset) s(ize) f(rag) F(size) q(uit)? s
  'b' size [0]: 480
  modify partition 'b'
  d(isplay) z(ero) t(ype) o(ffset) s(ize) f(rag) F(size) q(uit)? t
  'b' fstype [unused]: swap
  modify partition 'b'
  d(isplay) z(ero) t(ype) o(ffset) s(ize) f(rag) F(size) q(uit)? q
  modify partitions
  d(isplay) n(umber) s(elect) q(uit)? d
  
  type: old DEC
  disk: 
  label: DEFAULT
  flags:
  bytes/sector: 512
  sectors/track: 20
  tracks/cylinder: 2
  sectors/cylinder: 40
  cylinders: 512
  rpm: 3600
  drivedata: 0 0 0 0 0
  
  2 partitions:
  #        size   offset    fstype   [fsize bsize]
    a: 20000 0  2.11BSD    1024 1024      # (Cyl. 0 - 499)
    b: 480 20000  swap                    # (Cyl. 500 - 511)
  
  modify partitions
  d(isplay) n(umber) s(elect) q(uit)? q
  modify
  d(isplay) g(eometry) m(isc) p(artitions) q(uit)? q
  d(isplay) D(efault) m(odify) w(rite) q(uit)? w
  d(isplay) D(efault) m(odify) w(rite) q(uit)? q
  
  45Boot from vt(0,0,1) at 0177560
  : vt(0,0,2)
  
  Opened mkfs read-write
   rrrrrrrrrrrrrrrrrrrrrrrrrrrrrrrrrrrrrrrrrrrrrrBoot: bootdev=06402
                                                       bootcsr=0177560
  Mkfs
  file system: rl(0,0,0)
  file sys size [10000]: 
  bytes per inode [4096]: 
  interleaving factor (m; 2 default): 
  interleaving modulus (n; 20 default): 
  isize = 2496
  m/n = 2 20
  Exit called
  
  45Boot from vt(0,0,2) at 0177560
  : vt(0,0,4)
  
  Opened icheck read-write
   rrrrrrrrrrrrrrrrrrrrrrrrrrrrrrrrrrrrrrrrrrrrrrrrrrBoot: bootdev=06403
                                                           bootcsr=0177560
  Icheck
  File: rl(0,0,0)
  rl(0,0,0):
  files 3 (r=1,d=2,b=0,c=0,l=0,s=0)
  used 2 (i=0,ii=0,iii=0,d=2)
  free 9840
  missing 0
  
  45Boot from vt(0,0,4) at 0177560
  : vt(0,0,3)
  
  Opened restor read-write
   rrrrrrrrrrrrrrrrrrrrrrrrrrrrrrrrrrrrrrrrrrrrrrrrrrrrBoot: bootdev=06404
                                                             bootcsr=0177560
  Restor
  Tape? vt(0,0,5)
  
  Opened root.dump read-write
   Disk? rl(0,0,0)
  Last chance before scribbling on disk. rrrrrrrrrrrrr ... 
  rrrrrrrrrEnd of tape
 
  45Boot from vt(0,0,4) at 0177560
  : rl(0,0,0)unix
  Boot: bootdev=03400 bootcsr=0174400
  
  2.11 BSD UNIX #115: Sat Apr 22 19:07:25 PDT 2000
      sms1@curly.2bsd.com:/usr/src/sys/GENERIC
  [etc.]
\end{verbatim}

\section{Other Comments}

The {\tt vtserver/} directory holds the source code {\tt vtserver.c} as
well as the client binaries {\it boot.dd}, {\it copy}, {\it disklabel},
{\it mkfs}, {\it icheck} and {\it restor}.

The {\tt pdpvtstand/} directory holds the source code for the PDP-11 side of
things: {\tt vtboot.pdp}, {\tt boot}, {\tt copy}, {\it disklabel},
{\it mkfs}, {\it icheck} and {\it restor}. In fact, this is a copy
of {\tt /sys/pdpstand} from 2.11BSD which has been modified to have the
{\tt vt.c} client, the source code for {\tt copy} and {\tt vtboot.pdp}.

To compile all of this, you'll need a 2.11BSD system. Place
this directory in {\tt /usr/src/sys}, change into {\tt pdpvtstand/}
and run {\it make}. See the RCS comments for details.

\section{Caveats}

The VTserver software should be considered in beta release.
Your feedback is most definitely requested, to fix any bugs and to improve
the software. In particular, I haven't tried this software on all PDP-11
platforms: it should work on a system without split I\&D, but it does
require 192 Kbytes of memory.

More specifically, I've used the Ersatz-11 2.0 demo simulator with various
CPU models, and RL02 and RK05 disk images, to test {\it copy}. Here are the
results: {\it copy} can read and write disk images for /24, /34A, /40, /44,
/45, /70 and /94 systems when they have 256Kbytes of memory. It doesn't work
for the 11/35 as it doesn't have the MUL instruction, which the 2.11BSD C
compiler generates.

Also, I have never owned or used a real PDP-11. The VTserver software was
developed using the Ersatz-11 2.0 simulator and Bob Supnik's 2.3 simulator.
For this reason, there may be hardware-specific bugs in the software which
were not found using the simulators. As well, the terminology in this manual
may not be correct PDP-11 speak. Again, please send me your suggestions and
improvements!

\end{document}
